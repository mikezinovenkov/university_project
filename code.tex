\documentclass[12pt]{article}
\usepackage[utf8]{inputenc}
\usepackage[russian]{babel}
\usepackage{amsmath,amssymb}
\usepackage{graphics}
\usepackage{pbox}
\usepackage[x11names]{xcolor}
\definecolor{brightmaroon}{rgb}{0.76, 0.13, 0.28}
\definecolor{royalazure}{rgb}{0.0, 0.22, 0.66}
\usepackage[colorlinks=true,linkcolor=royalazure]{hyperref}
\usepackage{tikz, tkz-fct, pgfplots}
\usetikzlibrary{arrows}
\usepackage{geometry}
\geometry{
	a4paper,
	total={170mm,257mm},
	left=20mm,
	top=20mm
} 
\usepackage[labelsep=period]{caption}


% ----------------- Commands ----------------- 
\newcommand{\eps}{\varepsilon}
\newcommand\tline[2]{$\underset{\text{#1}}{\text{\underline{\hspace{#2}}}}$}

% ----------------- Set graphics path ----------------- 
\graphicspath{{img/}}

\begin{document}
\pagestyle{empty}
\centerline{\large Министерство науки и высшего образования}	
\centerline{\large Федеральное государственное бюджетное образовательное}
\centerline{\large учреждение высшего образования}
\centerline{\large ``Московский государственный технический университет}
\centerline{\large имени Н.Э. Баумана}
\centerline{\large (национальный исследовательский университет)''}
\centerline{\large (МГТУ им. Н.Э. Баумана)}
\hrule
\vspace{0.5cm}
\begin{figure}[h]
\center
\includegraphics[height=0.35\linewidth]{bmstu-logo-small.png}
\end{figure}
\begin{center}
	\large	
	\begin{tabular}{c}
		Факультет ``Фундаментальные науки'' \\
		Кафедра ``Высшая математика''		
	\end{tabular}
\end{center}
\vspace{0.5cm}
\begin{center}
	\LARGE \bf	
	\begin{tabular}{c}
		\textsc{Отчёт} \\
		по учебной практике \\
		за 3 семестр 2020---2021 гг.
	\end{tabular}
\end{center}
\vspace{0.5cm}
\begin{center}
	\large
	\begin{tabular}{p{5.3cm}ll}
		\pbox{5.45cm}{
			Руководитель практики,\\
			ст. преп. кафедры ФН1} 	& \tline{\it(подпись)}{5cm} & Кравченко О.В. \\[0.5cm]
		студент группы ФН1--31 		& \tline{\it(подпись)}{5cm} & Зиновенков М.В.
	\end{tabular}
\end{center}
\vfill
\begin{center}
	\large	
	\begin{tabular}{c}
		Москва, \\
		2020 г.
	\end{tabular}
\end{center}

\newpage	
\tableofcontents

\newpage
\section{Цели и задачи практики}	
\subsection{Цели}
--- развитие компетенций, способствующих успешному освоению материала бакалавриата и необходимых в будущей профессиональной деятельности.

\subsection{Задачи}
\begin{enumerate}
\item Знакомство с теорией рядов Фурье, и теорией интегральный уравнений.
\item Развитие умения поиска необходимой информации в специальной литературе и других источниках.
\item Развитие навыков составления отчётов и презентации результатов.
\end{enumerate}

\subsection{Индивидуальное задание}	
\begin{enumerate}
\item Изучить способы отображения математической информации в системе вёртски \LaTeX.
\item Изучить возможности  системы контроля версий \textsf{Git}.
\item Научиться верстать математические тексты, содержащие формулы и графики в системе \LaTeX.
Для этого, выполнить установку свободно распространяемого дистрибутива \textsf{TeXLive} и оболочки \textsf{TeXStudio}.
\item Оформить в системе \LaTeX типовые расчёты по курсе математического анализа согласно своему варианту.
\item Создать аккаунт на онлайн ресурсе \textsf{GitHub} и загрузить исходные \textsf{tex}--файлы 
и результат компиляции в формате \textsf{pdf}.
\item Решить индивидуальное домашнее задание согласно своему варианту, и оформить решение с учётов пп. 1---4.
\end{enumerate} 

\newpage
\section{Отчёт}
Интегральные уравнения имеют большое прикладное значение, являясь мощным
орудием исследования многих задач естествознания и техники: они широко используются
в механике, астрономии, физике, во многих задачах химии и биологии. Теория линейных
интегральных уравнений представляет собой важный раздел современной математики,
имеющий широкие приложения в теории дифференциальных уравнений, математической
физике, в задачах естествознания и техники. Отсюда владение методами теории
дифференциальных и интегральных уравнений необходимо приклажному математику, при решении задач
механики и физики.

\newpage
\section{Индивидуальное задание}
%\subsection{Элементарные функции и их графики.}
%\input{src/part1.tex}

%=================================================================================================================================
\subsection{Ряды Фурье и интегральное уравнение Вольтерры.}



% ---------------------------- Problem 1----------------------------------
\subsubsection*{\center Задача № 1.}
{\bf Условие.~}
Разложить в ряд Фурье заданную функцию $f(x)$, построить графики $f(x)$ и суммы ее ряда Фурье. Если не указывается, какой вид разложения в ряд необходимо представить, то требуетчя разложить функцию либо в общий тригонометрический ряд Фурье, либо следует выбрать оптимальный вид разложения в зависимости от данной функции.

\begin{center}
$f(x)=\frac{x-\pi}{2}$, 	& $0 \leqslant x \leqslant \pi$, & по косинусам.
\end{center}\\
{\bf Решение.~}	
%График
\begin{center}
\begin{minipage}[t][7cm][c]{7cm}
		\begin{tikzpicture}[
		declare function={
			func(\x)=
			and(\x >= 0, \x <= 1.5708) * 0.63662 * (1.5708 - \x) + 
			and(\x >  1.5708, \x <= 3.1416) * 0.63662 * (\x - 1.5708);
		}
		]
		\begin{axis}[
		axis x line=middle, axis y line=middle,
		axis equal,	
		ymin=-1.1, ymax=1.1, ytick={-1.5708,-1,1,1.5708}, ylabel=$y$,
		xmin=-1.1, xmax=5, xtick={-1,1,1.5708,2,3.14159},
		xticklabels={-1,1,$\dfrac{\pi}{2}$,2,$\pi$},
		yticklabels={$-\dfrac{\pi}{2}$,-1,1,$\dfrac{\pi}{2}$},
		xlabel=$x$,
		domain=0.0:pi-0.01,samples=600 % added		
		]
		
		\addplot [domain=0:3.14, black,line width=2pt] {(\x/2 - 1.5708)};
		\end{axis}
		\end{tikzpicture}
	\end{minipage} \\
	\end{center}
\noindent
Построим общий тригонометрический ряд Фурье вида
$$
f(x)=\frac{a_0}{2}+\sum_{n=1}^\infty 
	\left(a_n\cos{(n\omega x)}\right),\quad\text{где }\,\omega=\frac{\pi}{T},\,T=\pi.
$$
\noindent
Вычислим коэффициенты
$$
\begin{array}{rcl}
a_0 &=& \displaystyle\frac{2}{\pi}\left
\left(
\int\limits_0^\pi
\frac{x-\pi}{2}\,dx \right) 
 = -\frac{\pi}{2},												\\[12pt]
a_n &=& \displaystyle\frac{2}{\pi}\left(
	-\int\limits_0^\pi
	\frac{\pi}{2}\cos (nx)\,dx + \int\limits_0^\pi
	\frac{x}{2}\cos (nx)\,dx \right) ={}									\\[12pt]
	&=& \displaystyle \frac{2}{\pi}\left(
	-\pi*\left.\frac{\sin(nx)}{2n} \right|_0^\pi
	+\left.\frac{x\sin(nx)}{2 n} \right|_0^\pi 
	+\left.\frac{\cos(nx)}{2 n^2} \right|_0^\pi\right) = 	\\[12pt]
	&=& \displaystyle\frac{2}{\pi}(\frac{\cos(\pi n)}{2n^2}-(\frac{1}{2n^2})) \\[12pt]
	&=& \displaystyle\frac{(-1)^n-1}{\pi n^2},	\\[12pt]
\end{array}
$$
Применив теорему Дирихле о поточечной сходимости ряда Фурье, видим, что построенный ряд Фурье сходится 
к периодическому (с периодом $T=\pi$) продолжению исходной функции при всех $x$.
График функции $S(x)$ имеет следующий вид
\begin{center}
\begin{minipage}[t][7cm][c]{7cm}
		\begin{tikzpicture}[
		declare function={
			func(\x)=
			and(\x >= 0, \x <= 1.5708) * 0.63662 * (1.5708 - \x) + 
			and(\x >  1.5708, \x <= 3.1416) * 0.63662 * (\x - 1.5708);
		}
		]
		\begin{axis}[
		axis x line=middle, axis y line=middle,
		axis equal,	
		ymin=-1.1, ymax=1.1, ytick={-1.5708,-1,1,1.5708}, ylabel=$y$,
		xmin=-5.5, xmax=5.5, xtick={-3.14159,-2,-1.5708,-1,1,1.5708,2,3.14159},
		xticklabels={$-\pi$,-2,$-\dfrac{\pi}{2}$,-1,1,$\dfrac{\pi}{2}$,2,$\pi$},
		yticklabels={$-\dfrac{\pi}{2}$,-1,1,$\dfrac{\pi}{2}$},
		xlabel=$x$,
		domain=0.0:pi-0.01,samples=600 % added		
		]
		
		\addplot [domain=0:3.14, black,line width=2pt] {(\x/2 - 1.5708)};
		\addplot [domain=-3.14:0, black,line width=2pt] {(-\x/2 - 1.5708)};
		
		\addplot [domain=-6.28:-3.14, black,line width=2pt] {(\x/2 + 1.5708)};
		\addplot [domain=3.14:6.28, black,line width=2pt] {(-\x/2 + 1.5708)};
		
		\end{axis}
		\end{tikzpicture}
	\end{minipage} \\
	\end{center}
\noindent
\textbf{Ответ:}
\[
\begin{split}
f(x)=-\frac{\pi}{4}+\sum_{n=1}^\infty 
	\left((\frac{(-1)^n-1}{\pi n^2}\cos{(n x)}\right)
\end{split}
\]




% ---------------------------- Problem 2----------------------------------
\subsubsection*{\center Задача № 2.}
{\bf Условие.~}
Для заданной графически функции $y(x)$ построить ряд Фурье в комплексной форме, изобразить график суммы построенного ряда

%График
\begin{center}
	\begin{minipage}[t][7cm][c]{7cm}
		\begin{tikzpicture}[
		declare function={
			func(\x)=
			and(\x >= 0, \x <= 1.5708) * (1-cos(deg(\x))) + 
			and(\x >  1.5708, \x <= 3) * 0.0;
		}
		]
		\begin{axis}[
		axis x line=middle, axis y line=middle,
		axis equal,	
		ymin=-1.1, ymax=1.1, ytick={-1,...,1}, ylabel=$y$,
		xmin=-1.1, xmax=5, xtick={-1,1,1.5708,2,3.14159},
		xticklabels={-1,1,$\dfrac{\pi}{2}$,2,$\pi$},
		xlabel=$x$,
		domain=0.0:pi,samples=600 % added		
		]
		
		\addplot [domain=0:1.5708,black,line width=2pt] {1-cos(deg(\x))};
		\addplot [domain=1.5708:pi,black,line width=2pt] {0.0};
		\addplot [dashed, black] coordinates {(1.5708,0)(1.5708,1)};
		\node[] at (axis cs: 1.00,1.25) {\small$1-\cos{x}$};							
		\end{axis}
		\end{tikzpicture}
	\end{minipage} \\[1.5cm]
\end{center}

\noindent
\textbf{Решение.}\\

\noindent
Ряд Фурье в комплексной форме имеет следующий вид
\[
f(x) = \sum_{n=-\infty}^\infty c_n e^{i\omega nx},\quad c_n=\frac{1}{T}\int\limits_a^b f(x) e^{-i\omega nx}dx,\,\omega=\frac{2\pi}{T}.
\]
В нашем примере $ a=0,b=3,T=\pi,\omega=2$, 
найдем коэффицинеты $c_n,\,n=0,\pm1,\pm2,\ldots$
где $\omega=2,\,T=\pi.$
$$
\begin{array}{rcl}
c_0 &=&\displaystyle\frac{1}{\pi} \int\limits_0^\pi f(x)dx=\frac{a_0}{2}=\frac{\pi+2}{4\pi},\\[12pt]
a_n &=&\displaystyle\frac{1}{\pi}\left(\frac{\sin(\frac{\pi n}{2})}{n} + \frac{\cos(\frac{\pi n}{2})}{n^2-1}\right),\\[12pt]
b_n &=&\displaystyle\frac{1}{\pi}\left(\frac{1-\cos(\frac{\pi n}{2})}{n} + \frac{n - \sin(\frac{\pi n}{2})}{1-n^2}\right),\\[12pt]
c_n &=&\displaystyle\frac{a_n - i b_n}{2} = {}\\[12pt]
&=&\displaystyle\frac{1}{2\pi}\left(
\frac{\sin(\frac{\pi n}{2})}{n} + \frac{\cos(\frac{\pi n}{2})}{n^2-1}
- i \left(\frac{1 - cos(\frac{\pi n}{2})}{n} + \frac{n - \sin(\frac{\pi n}{2})}{1 - n^2}\right) \right)\\[12pt]
\end{array}
$$
\noindent
Применив теорему Дирихле о поточечной сходимости ряда Фурье, видим, что построенный ряд Фурье сходится 
к периодическому (с периодом $T=\pi$) продолжению исходной функции при всех $x\ne \frac{\pi n}{2}$, и $S(\frac{\pi n}{2})=1/2$ при 
$n=0,\pm1,\pm2,\ldots$, где $S(x)$ --- сумма ряда Фурье. График функции $S(x)$ имеет вид
\begin{center}
	\begin{tikzpicture}[
		declare function={
			func(\x)=
			and(\x >= 0, \x <= 1.5708) * (1-cos(deg(\x))) + 
			and(\x >  1.5708, \x <= 3) * 0.0;
		}
		]
		\begin{axis}[
		axis x line=middle, axis y line=middle,
		axis equal,	
		ymin=-1.1, ymax=1.1, ytick={-1,...,1}, ylabel=$y$,
		xmin=-5, xmax=5, xtick={-3.14159,-2,-1.5708,-1,1,1.5708,2,3.14159},
		xticklabels={$-\pi$,-2,$-\dfrac{\pi}{2}$,-1,1,$\dfrac{\pi}{2}$,2,$\pi$},
		xlabel=$x$,
		domain=0.0:pi,samples=600 % added		
		]
		
		\addplot [domain=0:1.5708,black,line width=2pt] {1-cos(deg(\x))};
		\addplot [domain=1.5708:pi,black,line width=2pt] {0.0};
		\addplot [dashed, black] coordinates {(1.5708,0)(1.5708,1)};
		
		\addplot [domain=-1.5708:0,black,line width=2pt] {0.0};
		\addplot [domain=-pi:-1.5708,black,line width=2pt] {1-cos(deg(\x+3.14))};
		\addplot [dashed, black] coordinates {(-1.5708,0)(-1.5708,1)};
		
		\addplot[
	mark=*,
	mark options={fill=black, draw=black},
	only marks,
	] coordinates {(1.57, 0.5)(-1.57, 0.5)};
	
		\node[] at (axis cs: 1.00,1.25) {\small$1-\cos{x}$};						
	\end{axis}
	\end{tikzpicture}
\end{center}

\noindent
\textbf{Ответ:}
\[
\begin{split}
&f(x)=\sum_{n=-\infty}^\infty\left[ \frac{1}{2\pi}\left(
\frac{\sin(\frac{\pi n}{2})}{n} + \frac{\cos(\frac{\pi n}{2})}{n^2-1}
- i \left(\frac{1 - cos(\frac{\pi n}{2})}{n} + \frac{n - \sin(\frac{\pi n}{2})}{1 - n^2}\right) \right) \right] e^{i2 nx},~ x\ne \frac{\pi n}{2}; \\
&S(\frac{\pi n}{2})=\frac{1}{2},\quad\text{при}~n\in\mathbb{Z}.
\end{split}
\]


% ---------------------------- Problem 3----------------------------------
\subsubsection*{\center Задача № 3.}
{\bf Условие.~}\\
Найти резольвенту для интегрального уравнения Вольтерры со следующим ядром 
$$K(x,t)=(t-x)7^{(\sin{t}-\sin{x})}, \lambda = 49.$$

\noindent
{\bf Решение.~}\\
\noindent
Из рекурентных соотношений $K_j(x,t)=\int\limits_t^x K(x,s)K_{j-1}(s,t)ds$ получаем
$$
\begin{array}{rcl}
K_1(x,t)&=&\displaystyle (t-x)7^{(\sin{t}-\sin{x})}, \\[12pt]
K_2(x,t)&=&\displaystyle\int\limits_t^x K(x,s)K_1(s,t)ds = \int\limits_t^x (s-x)7^{(\sin{s}-\sin{x})}
(t-s)7^{(\sin{t}-\sin{s})} ds = {}\\[12pt]
&=&\displaystyle 7^{(\sin{t}-\sin{x})}\cdot \frac{x^3-3x^2 t + 3xt^2 -t^3}{6} = 7^{(\sin{t}-\sin{x})}\cdot \frac{(x-t)^3}{6},\\[12pt]
K_3(x,t)&=&\displaystyle\int\limits_t^x K(x,s)K_2(s,t)ds = \int\limits_t^x (s-x)7^{(\sin{s}-\sin{x})} 7^{(\sin{t}-\sin{s})}\cdot \frac{(s-t)^3}{6}  ds = {}\\[12pt]
&=&\displaystyle 7^{(\sin{t}-\sin{x})}\cdot \frac{(t-x)^5}{120}.\\[12pt]
K_j(x,t)&=&\displaystyle \frac{(t-x)^{j+2}\cdot (-1)^{j+1}\cdot 7^{\sin(t)-\sin(x)}}{(2j-1)!},\quad j=\mathbb{N}.
\end{array}
$$
Подставляя это выражение для итерированных ядер, найдем резольвенту
$$ 
R(x,t,\lambda)=\sum_{j=1}^\infty \lambda^{j-1}K_j(x,t) = \sum_{j=1}^\infty 49^{j-1} \frac{(t-x)^{j+2}\cdot (-1)^{j+1} \cdot7^{\sin(t)-\sin(x)}}{(2j-1)!}
\quad j=1,2,\ldots
$$


\newpage
\addcontentsline{toc}{section}{Список литературы}
\begin{thebibliography}{99}
\bibitem{book01} Львовский С.М. Набор и вёрстка в системе \LaTeX,\,2003.
\bibitem{book02} Краснов М.Л., Киселев А.И., Макаренко Г.И. Интегральные уравнения. М.:~Наука,\,1976.
\bibitem{book03} Васильева А. Б., Тихонов Н. А. Интегральные уравнения. --- 2-е изд., стереотип. --- М:~ФИЗМАТЛИТ,\,2002.
\end{thebibliography}

\end{document}
